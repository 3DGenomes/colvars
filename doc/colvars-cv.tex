Collective variables and biases can be added, queried and deleted through the scripting command \texttt{cv}, with the following syntax: \texttt{cv~$<$subcommand$>$ [args...]}.
For example, to query the value of a collective variable named \texttt{myVar},
use the following syntax: \texttt{set value [cv colvar myVar value]}.
All subcommands of \texttt{cv} are documented below.

% The NAMD and VMD manuals have this section at different levels
\newcommand{\cvparagraph}[2]{\cvsubsec{#1}{#2}}
\cvvmdonly{
\renewcommand{\cvparagraph}[2]{\cvsubsubsec{#1}{#2}}
}

\cvparagraph{Managing the colvars module}{sec:cv_command_general}

\begin{itemize}
\cvvmdonly{
\item \texttt{molid} \emph{$<$molid$>$}: setup the colvars module for the given molecule; \emph{$<$molid$>$} is the identifier of a valid VMD molecule, either an integer or \texttt{top} for the top molecule;
\item \texttt{delete}: delete the colvars module; after this, the module can be setup again using \texttt{cv molid};
}
\item \texttt{configfile} \emph{$<$file name$>$}: read configuration from a file;
\item \texttt{config} \emph{$<$string$>$}: read configuration from the given string; both \texttt{config} and \texttt{configfile} subcommands may be invoked multiple times;
\cvvmdonly{see \ref{sec:colvars_config} for the configuration syntax;}
\item \texttt{reset}: delete all internal configuration of the colvars module;
\item \texttt{version}: return the version of the colvars code.
\end{itemize}

\cvparagraph{Input and output}{sec:cv_command_io}

\begin{itemize}
\item \texttt{list}: return a list of all currently defined variables;
\cvvmdonly{See \ref{sec:colvar}, \ref{sec:cvc} and \ref{sec:colvar_atom_groups} for how to configure a collective variable;}
\item \texttt{list biases}: return a list of all currently defined biases (i.e.~sampling and analysis algorithms);
\item \texttt{load} \emph{$<$file name$>$}: load a collective variables state file, typically produced during a previous simulation; \cvvmdonly{this parameter requires that the corresponding configuration has already been loaded by \texttt{configfile} or \texttt{config}; see \ref{sec:colvars_output} for a brief description of this file; the contents of this file are not required for as long as the VMD molecule has valid coordinates and \texttt{cv update} is used;}
\item \texttt{load} \emph{$<$prefix$>$}: same as above, but without the \texttt{.colvars.state} suffix;
\item \texttt{save} \emph{$<$prefix$>$}: save the current state in a file whose name begins with the given argument; if any of the biases have additional output files defined, those are saved as well using the same prefix;
\item \texttt{update}: recalculate all colvars and biases based on the current atomic coordinates;
\item \texttt{addenergy} \emph{$<$E$>$}: add value \emph{E} to the total bias energy; this must be used within \texttt{calc\_colvar\_forces};
\item \texttt{printframe}: return a summary of the current frame, in a format equivalent to a line of the collective variables trajectory file;
\item \texttt{printframelabels}: return text labels for the columns of \texttt{printframe}'s output;
\cvvmdonly{
\item \texttt{frame}: return the current frame number, from which colvar values are calculated;
\item \texttt{frame} \emph{$<$new frame$>$}: set the frame number; returns 0 if successful, nonzero if the requested frame does not exist.
}
\end{itemize}

\cvparagraph{Accessing collective variables}{sec:cv_command_colvar}

\begin{itemize}
\item \texttt{colvar} \emph{$<$name$>$} \texttt{value}: return the current value of colvar \emph{$<$name$>$};
\item \texttt{colvar} \emph{$<$name$>$} \texttt{update}: recalculate colvar \emph{$<$name$>$};
\item \texttt{colvar} \emph{$<$name$>$} \texttt{type}: return the type of colvar \emph{$<$name$>$};
\item \texttt{colvar} \emph{$<$name$>$} \texttt{delete}: delete colvar \emph{$<$name$>$};
\cvnamdonly{
\item \texttt{colvar} \emph{$<$name$>$} \texttt{addforce} \emph{$<$F$>$}: apply given force on colvar \emph{$<$name$>$};
\item \texttt{colvar} \emph{$<$name$>$} \texttt{getappliedforce}: get the force currently being applied on colvar \emph{$<$name$>$};
\item \texttt{colvar} \emph{$<$name$>$} \texttt{gettotalforce}: get the total force acting on colvar \emph{$<$name$>$} at the previous step (see \ref{sec:cvc_sys_forces});
}
\item \texttt{colvar} \emph{$<$name$>$} \texttt{getconfig}: return config string of colvar \emph{$<$name$>$}.
\item \texttt{colvar} \emph{$<$name$>$} \texttt{cvcflags} \emph{$<$flags$>$}: for a colvar with several cvcs (numbered according to their name
string order), set which cvcs are enabled or disabled in subsequent evaluations according to a list of 0/1 flags (one per cvc).
\end{itemize}

\cvparagraph{Accessing biases}{sec:cv_command_bias}

\begin{itemize}
\item \texttt{bias} \emph{$<$name$>$} \texttt{energy}: return the current energy of the bias \emph{$<$name$>$};
\item \texttt{bias} \emph{$<$name$>$} \texttt{update}: recalculate the bias \emph{$<$name$>$};
\item \texttt{bias} \emph{$<$name$>$} \texttt{delete}: delete the bias \emph{$<$name$>$};
\item \texttt{bias} \emph{$<$name$>$} \texttt{getconfig}: return config string of bias \emph{$<$name$>$}.
\end{itemize}
