
% compiling the NAMD User's Guide

% define missing macros
\newcommand{\MDENGINE}{NAMD}
\newcommand{\cvrefmanonly}[1]{}
\newcommand{\cvvmdonly}[1]{}
\newcommand{\cvvmdugonly}[1]{}
\newcommand{\cvnamdonly}[1]{#1}
\newcommand{\cvnamdugonly}[1]{#1}
\newcommand{\cvnamebasedonly}[1]{#1}
\newcommand{\cvlammpsonly}[1]{}
\newcommand{\cvscriptonly}[1]{#1}
\newcommand{\cvleptononly}[1]{#1}
\newcommand{\cvcommand}{cv}
\newcommand{\cvurl}[1]{\url{#1}}

% use the NAMD UG macros to document keywords
\newcommand{\key}[5]{\NAMDCOLVARCONF{#1}{#2}{#3}{#4}{#5}}
\newcommand{\keydef}[6]{\NAMDCOLVARCONFWDEF{#1}{#2}{#3}{#4}{#5}{#6}}
\newcommand{\labelkey}[1]{\label{#1}}
\newcommand{\refkey}[2]{``\texttt{#1}'' (see \ref{#2})}
\newcommand{\formatkey}[1]{``\texttt{#1}''}
\newcommand{\dupkey}[4]{%
  \index{#2!\texttt{#1}}
  {\bf \large \tt #1:} see definition of \texttt{#1} in sec.~\ref{#3} (#4)%
}
\newcommand{\simkey}[3]{%
  \index{#2!\texttt{#1}}
  {\bf \large \tt #1:} analogous to \texttt{#3}%
}

\newcommand{\cvsec}[2]{\subsection{#1}\label{#2}}
\newcommand{\cvsubsec}[2]{\subsubsection{#1}\label{#2}}
\newcommand{\cvsubsubsec}[2]{\paragraph*{#1}\label{#2}}

\newcommand{\outputName}{\emph{outputName}}
\newcommand{\restartName}{\emph{restartName}}
\newcommand{\inputName}{\emph{inputName}}


